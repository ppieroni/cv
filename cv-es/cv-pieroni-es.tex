% LaTeX Curriculum Vitae Template
%
% Copyright (C) 2004-2009 Jason Blevins <jrblevin@sdf.lonestar.org>
% http://jblevins.org/projects/cv-template/
%
% You may use use this document as a template to create your own CV
% and you may redistribute the source code freely. No attribution is
% required in any resulting documents. I do ask that you please leave
% this notice and the above URL in the source code if you choose to
% redistribute this file.

\documentclass[letterpaper]{article}

\usepackage{hyperref}
\usepackage{geometry}

% Comment the following lines to use the default Computer Modern font
% instead of the Palatino font provided by the mathpazo package.
% Remove the 'osf' bit if you don't like the old style figures.
\usepackage[T1]{fontenc}
\usepackage[sc,osf]{mathpazo}

\usepackage[spanish,activeacute]{babel}
\usepackage{ucs}
\usepackage[utf8x]{inputenc}
\usepackage{amssymb}

% Set your name here
\def\name{Pablo E. Pieroni}

% Replace this with a link to your CV if you like, or set it empty
% (as in \def\footerlink{}) to remove the link in the footer:
\def\footerlink{http://jblevins.org/projects/cv-template/}

% The following metadata will show up in the PDF properties
\hypersetup{
  colorlinks = true,
  urlcolor = black,
  pdfauthor = {\name},
  pdfkeywords = {},
  pdftitle = {\name: Curriculum Vitae},
  pdfsubject = {Curriculum Vitae},
  pdfpagemode = UseNone
}

\geometry{
  body={6.5in, 8.5in},
  left=1.0in,
  top=1.25in
}

% Customize page headers
\pagestyle{myheadings}
\markright{\name}
\thispagestyle{empty}

% Custom section fonts
\usepackage{sectsty}
\sectionfont{\rmfamily\mdseries\Large}
\subsectionfont{\rmfamily\mdseries\itshape\large}

% Other possible font commands include:
% \ttfamily for teletype,
% \sffamily for sans serif,
% \bfseries for bold,
% \scshape for small caps,
% \normalsize, \large, \Large, \LARGE sizes.

% Don't indent paragraphs.
\setlength\parindent{0em}

% Make lists without bullets
\renewenvironment{itemize}{
  \begin{list}{}{
    \setlength{\leftmargin}{1.5em}
  }
}{
  \end{list}
}

\begin{document}

% Place name at left
\begin{flushright}
{\huge \name}\\
   358 Balboa, CABA, Argentina\\
   +54 11 45538091 \\
   +54 9 11 55842634 (local: 1555842634)\\
   pepieroni@gmail.com
\end{flushright}
% Alternatively, print name centered and bold:
%\centerline{\huge \bf \name}

\vspace{0.25in}

\section*{INFORMACIÓN PERSONAL}
   Fecha de nacimiento: 03-30-1986\\
   Lugar de nacimiento: Ciudad de Buenos Aires, Argentina\\
   Nacionalidad: argentina\\
   Sexo: masculino
 
\section*{EDUCACIÓN}
Actualmente soy estudiante de doctorado en física en la Universidad de Buenos Aires y miembro de la Colaboración Pierre Auger. 
Mi tema de trabajo es la búsqueda de neutrinos cósmicos ultra energéticos. 
Formo parte del experimento desde el año 2010.
\begin{itemize}
 \item \textbf{Estudiante de doctorado en física:} Facultad de Ciencias Exactas y Naturales, Universidad de Buenos Aires (2011).
 \item \textbf{Directores:}
 \begin{itemize}
  \item Prof. Ricardo Piegaia - Universidad de Buenos Aires (aia@df.uba.ar) 
  \item Prof. Jaime Alvarez-Mu\~niz - Universidad de Santiago de Compostela (jaime.alvarezmuniz@gmail.com)
 \end{itemize}
 \item \textbf{Fecha estimada de finalización: Mar 2015}
\end{itemize}
 
% \section*{EDUCATION}

Mis estudios universitarios comenzaron en 2006 en la Universidad de Buenos Aires donde obtuve mi licenciatura en ciencias físicas en 2011, dentro de la Colaboración Pierre Auger y a través del grupo de la UBA dirigido por el Profesor Ricardo Piegaia.

\begin{itemize}
 \item \textbf{Licenciatura en ciencias físicas:} ``Medición del flujo difuso de neutrinos ultra energéticos mediante lluvias atmosféricas rasantes con el Observatorio Pierre Auger''.
 Dirigido por el Profesor Ricardo Piegaia, Universidad de Buenos Aires (2011).
\end{itemize}
 
\section*{APTITUDES}

% As PhD student, my daily work consists in analyzing big amounts of data. I have strong experience in several object oriented and procedural programing languages such as C++, python and bash, as well as in linux, advanced text processing (sed, awk, grep, etc.), \LaTeX, root, R, etc.
% I also have experience applying multivariate methods and statistics analysis in general.

Cuento con vasta experiencia analizando grandes volúmenes de datos y aplicándo métodos estadísticos. También poseo una formación fuerte en informática, que incluye varios lenguajes de programación como C++, Python y Bash, grid computing system (SGE), procesamiento de texto avanzado (sed, awk, grep, etc.), \LaTeX{} and Linux en general.


\section*{IDIOMAS}

Mi idioma nativo es el español y poseo un nivel de inglés avanzado, tanto oral como escrito.


\section*{BECAS} 
   \begin{itemize} %\itemsep -2pt  % reduce space between items
   \item $\blacktriangleright$ CONICET doctoral tipo II. 2014 to 2015. 
   \item $\blacktriangleright$ CONICET doctoral tipo I. 2011 to 2013. 
   \item $\blacktriangleright$ EPLANET con lugar de trabajo en el Departamento de Física de la Universidad de Santiago de Compostela, España. Enero - Agosto 2014.
   \item $\blacktriangleright$ EPLANET con lugar de trabajo en el Departamento de Física de la Universidad de Santiago de Compostela, España. Junio - Agosto 2012.
   \item $\blacktriangleright$ EPLANET con lugar de trabajo en el Departamento de Física de la Universidad de Santiago de Compostela, España. Octubre 2011.
 \end{itemize}

 
\section*{PUBLICACIONES EN REVISTAS CON REFERATO} 
\subsection*{Como miembro de la colaboración Pierre Auger}
Del grán numero de publicaciones de la Colaboración Pierre Auger\footnote{La lista completa de publicaciones puede encontrarse en: http://www.auger.org/technical\_info/}, estuve directamente involucrado en las siguientes:
\begin{itemize} %\itemsep -2pt

 \item{$\blacktriangleright$} Pierre Auger Collaboration. ``Ultra-High Energy Neutrinos at the Pierre Auger Observatory'', Advances in High Energy Physics, 2013 708680 (2013)
 
 \item{$\blacktriangleright$} Pierre Auger Collaboration. ``Search for point-like sources of ultra-high energy neutrinos at the Pierre Auger Observatory and improved limit on the diffuse flux of tau neutrinos '', Astrophysical Journal Letters 755 L4 (2012)
 
 \item{$\blacktriangleright$} Pierre Auger Collaboration. ``A search for ultra-high energy neutrinos in highly inclined events at the Pierre Auger Observatory'', Phys. Rev. D 84, 122005 (2011)
 
\end{itemize}

 
\section*{PROCEEDINGS Y CONFERENCIAS} 
\subsection*{Como miembro de la colaboración Pierre Auger}

De los 40 artículos presentados por la Colaboración Pierre Auger en el ICRC 2013 (Rio de Janeiro, Brasil), estuve directamente involucrado en el siguiente:
 \begin{itemize}%\itemsep -2pt
  \item{$\blacktriangleright$} P. Pieroni for the Pierre Auger Collaboration, ``Ultra high energy neutrinos at Pierre Auger Observatory''
 \end{itemize}

De los 40 artículos presentados por la Colaboración Pierre Auger en el ICRC 2011 (Beijing, China), estuve directamente involucrado en el siguiente:
 \begin{itemize}%\itemsep -2pt
  \item{$\blacktriangleright$} Y. Guardincerri for the Pierre Auger Collaboration, ``The Pierre Auger Observatory and ultra-high energy neutrinos: upper limits to the diffuse and point source fluxes''
 \end{itemize}


\subsection*{Presentaciones en reuniones de la colaboración}       
\begin{itemize}\itemsep -1pt
  \item{$\blacktriangleright$} J. Alvarez-Muñiz, S. Navas, R. Piegaia, P. Pieroni. ``$\nu$ Update''. Auger Collaboration Meeting, Malargüe, Argentina, 16 - 20 June 2014.
  \item{$\blacktriangleright$} J. Alvarez-Muñiz, S. Navas, R. Piegaia, P. Pieroni. ``$\nu$ Task news''. Auger Collaboration Meeting, Malargüe, Argentina, 9 - 14 November 2014.
  \item{$\blacktriangleright$} J. Alvarez-Muñiz, S. Navas, R. Piegaia, P. Pieroni. ``News from the neutrino task: unblinding of new data period''. Auger Collaboration Meeting, Malargüe, Argentina, 9 - 14 November 2014.
  \item{$\blacktriangleright$} J. Alvarez-Muñiz, Y. Guardincerri, R. Piegaia, P. Pieroni. ``The old Earth-Skimming exposure calculation revisited''. Auger Collaboration Meeting, Malargüe, Argentina, 11 - 16 November 2013.
  \item{$\blacktriangleright$} J. Alvarez-Muñiz, Y. Guardincerri, R. Piegaia, P. Pieroni. ``The new Earth-Skimming analysis''. Auger Collaboration Meeting, Malargüe, Argentina, 11 - 16 November 2013.
  \item{$\blacktriangleright$} J. Alvarez-Muñiz, Y. Guardincerri, R. Piegaia, P. Pieroni. ``Status of the combined neutrino analysis''. Auger Collaboration Meeting, Malargüe, Argentina, 11 - 16 November 2013.
  \item{$\blacktriangleright$} J. Alvarez-Muñiz, Y. Guardincerri, R. Piegaia, P. Pieroni, J. Tiffenberg. ``ICRC PHNU2 - UHE neutrinos at the Pierre Auger Observatory''. Auger Analysis Meeting, Lisbon, Portugal, 3-7 June 2013.
  \item{$\blacktriangleright$} J. Alvarez-Muñiz, Y. Guardincerri, R. Piegaia, P. Pieroni, J. Tiffenberg. ``The combination of the three neutrino searches into a single one''. Auger Collaboration Meeting, Malargüe, Argentina, 23 February - 1 March 2013.
  \item{$\blacktriangleright$} J. Alvarez-Muñiz, Y. Guardincerri, R. Piegaia, P. Pieroni, J. Tiffenberg. ``The new Earth-Skimming analysis''. Auger Collaboration Meeting, Malargüe, Argentina, 23 February - 1 March 2013.
  \item{$\blacktriangleright$} J. Alvarez-Muñiz, Y. Guardincerri, R. Piegaia, P. Pieroni, J. Tiffenberg. ``Status of new Earth-skimming tau neutrino analysis''. Auger Collaboration Meeting, Malargüe, Argentina, 11-16 November 2012.
  \item{$\blacktriangleright$} J. Alvarez-Muñiz, Y. Guardincerri, R. Piegaia, P. Pieroni, J. Tiffenberg. ``Towards a new selection of Earth-skimming tau neutrinos''. Auger Analysis Meeting, Prague, Czech Republic, 18-22 June 2012.
  \item{$\blacktriangleright$} J. Alvarez-Muñiz, Y. Guardincerri, R. Piegaia, P. Pieroni, J. Tiffenberg. ``Estimation of the influence of $\nu_{\tau}$ interacting on the Andes to the point-source neutrino limit''. Auger Collaboration Meeting, Malargüe, Argentina, 13-18 November 2011.
  \item{$\blacktriangleright$} J. Alvarez-Muñiz, Y. Guardincerri, R. Piegaia, P. Pieroni, J. Tiffenberg. ``Limits to the UHE neutrino flux from Gamma-Ray Bursts with the SD of the Pierre Auger Observatory''. Auger Collaboration Meeting, Malargüe, Argentina, 13-18 November 2011.
  \item{$\blacktriangleright$} J. Alvarez-Muñiz, Y. Guardincerri, R. Piegaia,P. Pieroni, J. Tiffenberg. ``Summary and update on the down-going nu limit''. Auger Collaboration Meeting, Malargüe, Argentina, 13-19 November 2010.
 
\end{itemize}


\section*{DOCENCIA}

Desde 2010 tengo una posición como ayudante de profesor en el Departamento de Física de la Facultad de Ciencias Exactas y Naturales de la Universidad de Buenos Aires.


\end{document}
